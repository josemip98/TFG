\chapter{Planificación}

\section{Metodología utilizada}
Para el desarrollo del proyecto he elegido la \textbf{metodología SCRUM}.
Esta metodología es una de las más utilizadas en la actualidad y consiste en unas prácticas que permiten un trabajo de entregas que se va incrementando para el desarrollo de un producto. \\
La gran característica de esta metodología y por lo que recibe el nombre es su agilidad ya que consiste en presentar una serie de objetivos y/o requisitos necesarios, se les asigna una prioridad y se asigna al personal que va a trabajar en el proyecto.
Además otra característica es que el cliente va a poder empezar a utilizar el producto porque se va a ir desarrollando por partes por así decirlo.

Después se hace una planificación en la que se hace una estimación de los tiempos de entrega.
A partir de ahí estos requisitos que bien pueden ser historias de usuario, tareas de mantenimiento o correción de bugs del proyecto se van realizando.
 
\section{Seguimiento del desarrollo}

Para el seguimiento y gestión del desarrollo he utilizado el software de control de versiones
git en la plataforma Github. Esta plataforma me permite mediante un kanban organizar todo el proyecto
y tener un listado de requerimientos mediante historias de usuarios y tareas pendientes (issues) asignandoles
prioridad, asignandoselas a usuarios aunque en este caso todas estarán asignadas a mi ya que es un proyecto personal 
y más posibilidades.

Además para estar en contacto con mi tutor, he utilizado un Bot de Telegram que he conectado con mi repositorio y 
configurado para que cada commit o creación/cierre de issues se notifique en un grupo de telegram en el que estamos mi tutor y yo
junto con dicho Bot.

\subsection{Kanban}

Se ha creado un proyecto de GitHub y se ha utilizado una tabla Kanban para gestionar las historias de usuario y las issues del proyecto.

El Kanban del proyecto puede verse en el siguiente enlace:
\url{https://github.com/josemip98/TFG/projects/1}

\begin{figure}[H]
  \centering
  \noindent\makebox[\textwidth]{
    \includegraphics[scale=0.4]{kanban.png}}
  \caption{Tabla Kanban del proyecto.}
\end{figure}

\subsection{Hitos}

Los hitos o milestones se utilizan como agrupación de issues o pull requests englobando un grupo de problemas o problemas más grandes que resolver.
Por ejemplo yo para mi proyecto he creado un sólo milestone llamado aplicación web pero se podría crear otro que fuera aplicación móvil para en un futuro realizarla y en dichos hitos irían todas las issues relacionadas.

\begin{figure}[H]
	\centering
	\noindent\makebox[\textwidth]{
	  \includegraphics[scale=0.4]{hito.png}}
	\caption{Ejemplo épica del proyecto.}
  \end{figure}

\subsection{Historia de usuario}
Una historia de usuario es \textbf{una funcionalidad que el usuario espera}.
El modelo a seguir para la creación de historias de usuario es:

\textit{Como usuario/desarrollador quiero poder [funcionalidad] para [razón dicha funcionalidad].}

Además podemos añadir tareas que necesitamos cumplir para completar la historia de usuario o detalles técnicos.

\begin{figure}[H]
	\centering
	\noindent\makebox[\textwidth]{
	  \includegraphics[scale=0.4]{hu.png}}
	\caption{Ejemplo historia de usuario.}
  \end{figure}

\subsection{Issues}
Las issues pueden ser correción de bugs, tareas que sean necesarias para el desarrollo o mantenimiento del proyecto.

\begin{figure}[H]
	\centering
	\noindent\makebox[\textwidth]{
	  \includegraphics[scale=0.4]{issue.png}}
	\caption{Ejemplo de issue.}
  \end{figure}

\subsection{Etiquetas}

Las etiquetas son la forma de categorizar las issues pendientes ya sea asignandoles prioridad o describiendo si es una issue de documentación, si es un bug, historia de usuario...
Además podemos crear etiquetas a nuestro gusto según las necesidades que tengamos.

\begin{figure}[H]
	\centering
	\noindent\makebox[\textwidth]{
	\includegraphics[scale=0.4]{etiquetas.png}}
	\caption{Ejemplo de etiquetas.}
\end{figure}