\chapter{Implementación}

\section{Creación modelos base de datos}
\subsection{Usuario}
\subsection{Producto}
\subsection{Dieta}

\section{Creación formularios}
\subsection{Usuario}
\subsection{Producto}
\subsection{Dieta}

\section{URLs}

\section{Importación de datos}

\section{Usuarios}
\subsection{Inicio sesión}
\subsection{Cerrar sesión}
\subsection{Registro}
\subsection{Ver perfil}
\subsection{Modificar perfil}

\section{Productos}
\subsection{Listado de productos}
\subsection{Añadir producto}
\subsection{Consultar producto}
\subsection{Modificar producto}
\subsection{Eliminar producto}
\subsection{Paginador}
\subsection{Búsqueda}
\subsection{Búsqueda productos similares}

\section{Dietas}
\subsection{Mostrar dieta}
\subsection{Generar dieta}
\subsection{Crear dieta}
\subsection{Ver dieta}
\subsection{Exportar dieta a PDF}

\section{Tests}

\section{Despliegue}

Por último, quedaría desplegar la aplicación, para ello voy a utilziar heroku.
Heroku es una plataforma en la nube que nos permite desplegar aplicaciones web en cualuier lenguaje de programación.
\\\\
Además es muy sencillo, sólo tenemos que conectar nuestro repositorio de GitHub en el que tengamos el proyecto que queremos desplegar,
podemos configurarlo para que con cada commit se haga el despliegue automaticamente.
En mi caso, he utilizado python por tanto lo único que tengo que hacer es crear un archivo Procfile en el que incluiremos el nombre de la aplicación,
el archivo requirements.txt con las librerias que tengamos que instalar y configurar el settings.py, que en mi caso fué poner a false el modo debug, conectar la base de datos,
incluir libreria whitenoise para cargar archivos estáticos e incluir la url de despliegue para aceptar peticiones.
\\\\
Enlace de despligue:
\url{https://agile-thicket-17815.herokuapp.com/}