\chapter{Estado del arte}

Primero de todo necesitaba tener una base de datos de productos alimenticios de la que partir. Para ello estuve buscando datasets de código abierto y encontré una web muy interesante.
https://es.openfoodfacts.org/
\\\\
\textbf{Open food facts:} es una gran base de datos de productos alimenticios de código abierto que nos permite consultar de cualquier producto su información nutricional.
Relacionado con el proyecto me fué de gran ayuda ya que me permitió descargar sus datos en distintos formatos para su posterior limpieza e importación.
\\\\
Una vez hecho esto, investigué sobre otras aplicaciones que estuvieran relacionadas con la creación de dietas. Encontré \textbf{Intake24} y \textbf{Eatthismuch}. 
\\\\
\textbf{Intake24:} es un web de código abierto que genera dietas basandose en la alimentación que has tenido en las 24 horas pasadas.
\\\\
\textbf{Inconveniente:}
Me pareció interesante pero no se ajusta a lo que quería desarrollar en este proyecto ya que consultando gente relacionadas
con el mundo de la nutrición me comentaron que para crear una dieta tienes que tener muchas cosas más en cuenta y no sólo
la alimentación que has tenido en el día anterior ya que para dos personas, esa misma alimentación puede afectar de distinas maneras.
\\\\
Enlace a la web: https://intake24.co.uk/info/open-source
\\\\
Por otro lado, encontré \textbf{Eatthismuch}:
\\\\
\textbf{Eatthismuch} genera dietas personalizadas basadas en las comidas que quieras hacer al día y la ingesta de calorías que quieras.
\\\\
\textbf{Inconveniente:}
Es otro punto de vista pero volvemos al mismo problema. Tu puedes ingerir por ejemplo 3000 calorías en las que gran parte sean proteinas
y muy pocas de grasa y vas a tener un resultado diferente a si de esas 3000 calorías la mayor parte son grasas.
\\\\
Este sería el enlace a la web: https://www.eatthismuch.com/
\\\\
Tras este análisis de aplicaciones similares a nuestro objetivo me decidí por crear una base de datos de productos en la que puedas consultar
la información nutricional de cada uno, con la opción tanto de crear tu propia de dieta si tienes conocimiento nutricional y asignarsela a un usuario o de solicitar que 
se te genere una dieta en base a tus características y objetivos. Una vez esto podrias consultar la dieta y intercambiar productos que sean similares 
nutricionalmente hablando.

