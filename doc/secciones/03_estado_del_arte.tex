\chapter{Estado del arte} \label{sec:estado_del_arte}

Después de investigar sobre otras aplicaciones que estuvieran relacionadas con la generación de dietas. Encontré \textbf{Intake24} \cite{Intake24} y \textbf{Eatthismuch} \cite{Eatthismuch}. 
\\\\
\textbf{Intake24:} es un web de código abierto que genera dietas basándose en la alimentación que has tenido en las 24 horas pasadas.
La web te pide tus datos personales y después te va preguntando sobre las comidas del día anterior y sobre tus gustos personales.
Su uso es gratis para todos ya que fue un proyecto de investigación de la Universidad de Cambridge junto con la Universidad de Newcastle y 
la FSS (Food Standards Scotland).
\\\\
\textbf{Inconveniente:}
Esta web es bastante interesante, pero no se ajusta a lo que quería desarrollar en este proyecto, ya que para crear una dieta tienes que tener muchas cosas más en cuenta y no sólo
la alimentación que has tenido en el día anterior, ya que, para dos personas, esa misma alimentación puede afectar de distintas maneras,
ya sea entre otras cosas por su metabolismo o por su actividad física semanal.
\\\\
Por otro lado, he encontrado la web de \textbf{Eatthismuch}:
\\\\
\textbf{Eatthismuch} genera dietas personalizadas basadas en el número de comidas que quieras hacer al día y la ingesta de calorías que quieras.
Esta aplicación sobretodo está pensada para dietistas ya que les va a facilitar llevar el seguimiento de sus clientes creando planes específicos
para cada cliente, además de permitir mandar correos electrónicos y PDF con la marca de su empresa.
Es gratis los primeros 15 días, después tiene un precio de 79 dólares (66,99 euros) al mes si lo compras durante 3 meses o de 99 dólares (83,95 euros) si lo contratas mensualmente.
Incluye 10 cuentas de clientes, si quisiéramos más se paga 4 dólares (3.39 euros) más por cada uno adicional.\\
Si no eres dietista y simplemente quieres obtener tu dieta también puedes, escoges tu tipo dieta (mediterránea, vegana, vegetariano...), el número de comidas que realizas al día y las calorías que quieres consumir y listo.
Además, da la posibilidad de hacer la lista de la compra y organizar tu dieta con un calendario.
Esta aplicación es muy interesante en todos los aspectos y se asemeja mucho al proyecto que quiero realizar. Los clientes van a ser los mismos, dietistas que quieran llevar la dieta de sus clientes
y usuarios con menos conocimientos dietéticos que quieran obtener una dieta. Además de todas las ventajas de tener un calendario para organizarte, planes personalizados, capacidad de exportar dietas a PDF y mucho más.
\\\\
\textbf{Inconveniente:}

A la hora de generar dietas volvemos al mismo problema que con la otra aplicación. Con sólo la información del número de comidas diarias y la cantidad de calorías a ingerir
no se consigue una dieta óptima ya que tú puedes ingerir, por ejemplo, 3000 calorías en las que gran parte sean proteínas y muy pocas de grasa y vas a tener un resultado diferente a si de 
esas 3000 calorías la mayor parte son grasas, añadiendo los problemas anteriormente mencionados de personas con diferentes metabolismos, diferente actividad física semanal, sexo, etc.
\\\\ 
Además de estas dos aplicaciones relacionadas con la generación de dietas quiero mencionar otra web que, aunque no está relacionado con esto es muy interesante y me va a ser muy útil.
Para la realización del proyecto primero de todo necesitaba tener una base de datos de productos alimenticios de la que partir. Para ello estuve buscando datasets de código abierto y encontré la web de \textbf{Open Food Facts} \cite{OpenFoodFacts}.
\\\\
\textbf{Open food facts:} es una gran base de datos de productos alimenticios de código abierto que nos permite consultar de cualquier producto su información nutricional.
Esta base de datos es gratis y es un proyecto a nivel mundial en el que colaboran muchas personas y nosotros mismos podemos ayudar incluyendo productos.
Relacionado con el proyecto me va a ser de gran ayuda ya que me va a permitir descargar sus datos en distintos formatos para su posterior limpieza e importación.
\\\\
Tras este análisis de aplicaciones similares a nuestro objetivo me he decidido por crear una \textbf{aplicacion web} con una gran base de datos de productos en la que puedas consultar
la información nutricional de cada uno, con la opción tanto de crear tu propia de dieta si tienes conocimiento nutricional y poder asignársela a otros usuarios, o de solicitar que 
se te genere una dieta en base a tus características y objetivos. Además, dar la posibilidad de recomendar productos que sean similares entre sí, nutricionalmente hablando.
