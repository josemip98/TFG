\chapter{Estado del arte}

La motivación para decantarme por este proyecto fué que siempre he sido un chico muy delgado y me decidí a hacer dieta para poder ganar unos kilos mientras hacía deporte.

Para ello contraté un dietista y me mandó una dieta estricta en la que prácticamente comía lo mismo todos los días y si bien es cierto que obtuve buenos resultados, pensé 
en porqué no hacer una aplicación en la que poder consultar todo tipo de productos junto con su información nutricional y así poder intercambiar productos con productos que 
sean equivalentes nutricionalmente porque no me gustara algún producto o no tuviera en casa y no pudiera ir a comprarlo o simplemente por cambiar y no comer siempre lo mismo.

Busqué y encontré una web: https://es.openfoodfacts.org/

Open Food Facts es una base de datos de productos alimentarios de código abierto que cualquiera puede aprovecharlos.

Después para poder probar varias dietas busqué aplicaciones y encontré Intake24

https://intake24.co.uk/info/open-source

Intake24 es un sistema que genera dietas de código abierto que se autocompleta y se basa en la alimentación que has tenidoen las 24 horasa pasadas. 
El sistema en línea ofrece una calidad de datos similar a los recordatorios dirigidos por entrevistadores a un costo significativamente menor.

y Eatthismuch:

https://www.eatthismuch.com/

Eat This Much crea planes de comidas personalizados basados ​​en sus preferencias alimentarias, presupuesto y horario. Alcance sus metas nutricionales y 
de dieta con nuestra calculadora de calorías, planes de comidas semanales, listas de compras y más. Cree su plan de comidas aquí mismo en segundos.

Ante esto pensé en juntar ambas aplicaciones y crear una base de datos de productos en la que además tengas la opción de generar tu propia dieta e intercambiar
los productos on otros que sean nutricionalmente similares.