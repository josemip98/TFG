\chapter{Descripción del problema} \label{sec:problema}

Actualmente con la situación que estamos viviendo muchas personas, en las que me incluyo, hemos 
decidido practicar algún deporte o simplemente hacer ejercicio además de acompañarlo con una dieta 
para mantenernos saludables y en forma, debido al teletrabajo, clases online y confinamiento.
\\\\
En mi caso y en el de muchas personas hemos tenido que contratar y estar en contacto con un dietista que nos lleve la dieta 
cada mes además de tener que comprar productos que, o no podemos permitirnos, o simplemente no nos gustan.\\

Debido a este problema pensé que, una muy buena solución tanto para dietistas como para usuarios con menos conocimientos de 
nutrición, sería la creación de una web con la que podamos obtener nuestra dieta en base a nuestros objetivos. Aparte, los 
dietistas no serían sustituidos ya que podrían crear dietas basándose en sus conocimientos y ayudándose sobre una gran variedad 
de productos que contendría la base de datos y asignárselas a sus clientes.\\

Además de dar la posibilidad de poder consultar productos, la web también podría recomendar productos similares tanto para personas que no 
tengan ni idea de nutrición y quieran variar su alimentación en base a sus gustos o situación económica, como a los dietistas para que 
las dietas sean más variadas y completas porque al final los dietistas no son máquinas, utilizan los productos más comunes de su zona o de sus propios gustos
y siempre te encuentras los mismos alimentos (pollo, arroz, patatas...) en varios tipos de dietas, cuando seguramente haya muchos productos que
sean similares nutricionalmente hablando.
\\\\
Esto ayudaría además a un problema que varias personas a las que he consultado que tienen bastante conocimiento sobre nutrición como dietistas y culturistas me han comentado, 
y es que es muy complicado obtener una dieta basándote en tu peso, altura, sexo, objetivos o cantidad de calorías diarias, ya que según me explican,
cada persona es un mundo y todos los datos que tengamos de una persona son pocos, además que algunos datos como su metabolismo puede que ni ellos mismos lo sepan.
Por ejemplo, dos personas con el mismo peso, altura, ambos del mismo sexo y con el objetivo de ganar volumen unos alimentos les pueden funcionar mejor que otros.\\ \\

Con este proyecto podríamos generar dietas y una vez obtenidas, mediante la obtención de productos similares podríamos ir modificando dicha dieta e ir probando hasta obtener resultados
óptimos, y ya de paso, obtenemos conocimientos de nutrición que nunca vienen mal.