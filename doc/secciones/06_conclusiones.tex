\chapter{Conclusiones y trabajos futuros} \label{sec:conclusion}

Como conclusión mencionar que los objetivos del proyecto se han cumplido en gran parte 
ya que hemos conseguido crear una aplicación web que facilite a todo tipo de usuarios, tanto a dietistas, como a usuarios con menos conocimientos, 
a mejorar sus conocimientos acerca de la nutrición, facilitarles una dieta en base a sus objetivos y optimizarlas.  \\

Me gustaría conseguir el objetivo de que esta aplicación le sirviera a mucha gente para encontrar una dieta que se ajuste
a sus necesidades y posibilidades, para así poder ayudar con el gran problema de los malos hábitos alimenticios.\\

En cuanto a mi, el proyecto me ha permitido aprender mucho como programador web en muchos ámbitos desde
la planificación con GitHub mediante épicas, historias de usuario e issues que es como se 
trabaja en las empresas, las tecnologías actuales de desarrollo web como Python, Django, 
Postgres, Bootstrap, HTML, CSS y algunas más, que son las tecnologías que he utilizado en 
este proyecto además de las tecnologías probadas antes de la elección de las mencionadas 
anteriormente como por ejemplo Kotlin para el desarrollo de aplicaciones móviles, Angular, 
React, PHP, Laravel, etc.
\\\\
Además, sobre el proyecto, era un tema que me interesaba bastante por lo que me ha sido más fácil 
trabajar sobre ello y me gustaría continuarlo en un futuro añadiendo más funcionalidad a 
la aplicación web como pudiera ser un calendario para organizar las comidas diarias, mejorar 
los algoritmos de similitud de alimentos mediante investigación nutricional, expansión de la 
base de datos de productos, capacidad de hacer un seguimiento sobre el usuario en términos de 
peso, grasa, músculo, etc. e incluso meterme en temas de ejercicios y planes de gimnasio para 
así ya tener todo lo que necesitas para estar saludable y en forma. Todo eso y más ideas que puedan ir surgiendo.
\\\\
Por otro lado, aparte de las mejoras mencionadas anteriormente me gustaría realizar mediante
Big Data algoritmos de creación de dietas basándonos en el análisis de grandes cantidades de
productos ya que tanto el desarrollo web como el análisis de datos, Big Data y Machine Learning
me parecen muy interesantes.
\\\\
Por tanto, puedo decir que estoy muy contento con todo el aprendizaje que he tenido realizando 
el proyecto, ya que me ha ha ayudado a la hora de encontrar trabajo y conseguir experiencia laboral en este tema 
de la programación web que tanto me interesa.