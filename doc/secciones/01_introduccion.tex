\chapter{Introducción}

Este proyecto es software libre, y está liberado con la licencia \cite{gplv3}.

\section{Motivación}

\section{Alcance}

Este proyecto está pensado para poder llegar a una gran número de usuarios de todo el país. Lo que se pretende es que sean los propios usuarios los encargados de dar a conocer el proyecto. Sabemos que a veces el propio boca a boca es una de las maneras más eficaces de propagación, y en la época en la que nos situamos las personas suelen buscar recomendaciones sobre los productos que van a consumir o utilizar. Si tu como usuario estas contento con la aplicación se la recomendarás a más gente de tu entorno, lo que provocará no solo un aumento de usuarios, ya que estos se lo dirán a su vez a otros, con el fin de conseguir llegar al máximo número de personas posibles.

\section{Objetivos generales}

El objetivo principal de este proyecto es el desarrollo de una aplicación en la que los usuarios puedan gestionar su dieta personalizada.

Para llevar a cabo el proyecto se pondrá en práctica la metodología de desarrollo Scrum, la cual usaremos para la planificación y desarrollo del proyecto, con el fin de aprovechar las ventajas que esta nos ofrece.

