\chapter{Introducción}

Este proyecto es software libre, y está liberado con la licencia \cite{gplv3}.

Ante la situación actual que vivimos con el covid-19, muchas personas han decidido hacer deporte 
para así mantenerse en forma debido a los confinamientos, teletrabajo o para despejarse de la 
situación y no estar aislados en casa. Además muchas personas quieren acompañar este deporte con una dieta o simplemente quieren hacer 
dieta para mantenerse saludables.

\section{Motivación}

La motivación para hacer este proyecto fué que yo siempre he sido un chico muy delgado y que 
me ha gustado el deporte, decidí a hacer dieta para poder ganar unos kilos mientras hacía deporte.
Para ello contraté un dietista, el cual me mandó una dieta estricta en la que prácticamente comía lo mismo 
todos los días y si bien es cierto que obtuve buenos resultados, era dificil de mantener. \\

Debido a esta situación pensé que si plasmaramos estos conocimientos de los dietistas en una aplicación, con una base de datos 
muy grande de productos, que te generara una dieta en base a tus características y objetivos 
además de poder consultar todo tipo de productos junto con su información nutricional y así poder 
intercambiar productos con productos que sean equivalentes nutricionalmente, todo sería más cómodo tanto para
el usuario que simplemente tiene que consultar una web como para los dietistas que lo gestionarían todo desde ahí.
\\\\
Ya que para ello estamos obligados a tener un dietista contratado todos los meses y tener recursos para 
comprar los productos que nos mandan y muchas personas o no tienen recursos o simplemente preferirían
que hubiera una aplicación en la que ellos mismos pudieran obtener su dieta y modificarla según sus 
posibilidades o gustos.

\section{Objetivos}

Los objetivos a cumplir por la aplicación que vamos a desarrollar debe ser:\\

Por un lado dar servicio tanto a usuarios que necesiten una dieta para cumplir sus objetivos además de la posibilidad de intercambiar productos 
que sean similares y poder consultar todo tipo de productos.\\
Y a dietistas que ya tienen el conocimiento y quieran crear ellos las dietas y consultar información nutricional de productos.\\

Por tanto, los objetivos principales del proyecto serían los siguientes:

\begin{enumerate}
    \item
    Sobre los productos: 
    \begin{itemize}
      \item Crear, modificar y eliminar sus propios productos.
      \item Consultar productos con su información nutricional.
      \item Consultar productos similares.
    \end{itemize}
    \item
    Sobre las dietas: 
    \begin{itemize}
    \item Poder generar dietas en base a características y objetivos.
    \item Crear tus propias dietas seleccionando tú los productos.
    \item Consultar dieta.
        \begin{itemize}
        \item Consultar tu propia dieta si eres usuario normal.
        \item Consultar las dietas de los usuarios si eres dietista.
        \end{itemize}
    \item Exportar a PDF.
    \end{itemize}
\end{enumerate}