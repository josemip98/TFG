\chapter{Introducción}

Este proyecto es software libre, y está liberado con la licencia \cite{gplv3}.

Ante la situación actual que vivimos con el covid-19, muchas personas han decidido hacer deporte 
para así mantenerse en forma debido a los confinamientos, teletrabajo o para despejarse de la 
situación y no estar aislados en casa. Además muchas personas quieren acompañar este deporte con una dieta o simplemente quieren hacer 
dieta para mantenerse saludables.

\section{Motivación}

La motivación para hacer este proyecto fué que yo siempre he sido un chico muy delgado y que 
me ha gustado el deporte, decidí a hacer dieta para poder ganar unos kilos mientras hacía deporte.
Para ello contraté un dietista, el cual me mandó una dieta estricta en la que prácticamente comía lo mismo 
todos los días y si bien es cierto que obtuve buenos resultados, era dificil de mantener. \\

Debido a esta situación pensé que si plasmaramos estos conocimientos de los dietistas en una aplicación, con una base de datos 
muy grande de productos, que te generara una dieta en base a tus características y objetivos 
además de poder consultar todo tipo de productos junto con su información nutricional todo sería más cómodo tanto para
el usuario que simplemente tiene que consultar una web como para los dietistas que lo gestionarían todo desde ahí
y podrían obtener mucha información. \\ \\
\\\\
Ya que para tener una dieta estamos obligados a tener un dietista contratado todos los meses y tener recursos para 
comprar los productos que nos mandan y muchas personas o no tienen recursos o simplemente preferirían
que hubiera una aplicación en la que ellos mismos pudieran obtener su dieta y modificarla según sus 
posibilidades o gustos o simplemente porque es mucho más accesible.

\section{Objetivos}

Los objetivos a cumplir por el proyecto es el \textbf{diseño y desarrollo de una aplicación web} que por un lado de servicio tanto a dietistas
que ya tienen el conocimiento y quieran crear dietas, proporcionando información nutricional para así hacerlas más completas y ganar 
conocimiento.\\

Y por otro lado, dar soporte a usuarios con menos conocimiento de nutrición, con el fin de obtener sus propias dietas en base a sus características y objetivos,
haciendo más accesible el conseguir una dieta o información que si tienes que estar en contacto con un dietista y siguiendo unas pautas muy estrictas.\\

En definitiva, intentar conseguir que mucha más gente se inicie en el mundo de las dietas al hacerlo mucho 
más accesible, que obtengan conocimientos sobre nutrición y obtengan una dieta saludable y variada para mejorar su alimentación,
ya que la obesidad o los malos hábitos alimenticios son un problema y son cada vez más comunes en nuestra sociedad.
