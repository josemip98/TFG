\thispagestyle{empty}

\begin{center}
{\large\bfseries OrganizeUDiet \\ Generación de dietas optimizadas con fines concretos}\\
\end{center}
\begin{center}
José Miguel Pelegrina Pelegrina\\
\end{center}

\vspace{0.7cm}

\vspace{0.5cm}
\noindent{\textbf{Palabras clave}: \textit{Python, Django, Nutrición, Dieta, Software libre}\\\\
\vspace{0.7cm}
\noindent{\textbf{Resumen}\\

Ante la situación que vivimos actualmente con el teletrabajo y las clases online mucha gente ha empezado a hacer deporte y hacer dieta para
mantenerse saludables, además del gran problema que supone los malos hábitos alimenticios en nuestra sociedad y su aumento debido
especialmente al desconocimiento en temas de nutrición, he decidido realizar el siguiente proyecto que consiste en la realización de una 
aplicación web con la que toda la gente pudiera obtener una dieta optimizada según sus características y objetivos, además ya de paso ganar
información acerca de la nutrición.\\\\

Este proyecto ayudará tanto a personas que quieran obtener una dieta y poder modificarla en base a sus gustos, necesidades y capacidad económica,
como a nutricionistas que quieran crear sus propias dietas y aumentar sus conocimientos con una gran base de productos en la que podrá compararlos
o cambiarlos entre los que sean similares.

\cleardoublepage

\begin{center}
	{\large\bfseries OrganizeUDiet \\ Generation of optimized diets for specific purposes}\\
\end{center}
\begin{center}
	José Miguel Pelegrina Pelegrina\\
\end{center}
\vspace{0.5cm}
\noindent{\textbf{Keywords}: \textit{Python, Django, Nutrition, Diet, Open source}
\vspace{0.7cm}

\noindent{\textbf{Abstract}\\ \\

Given the current situation with teleworking and online classes, many people have started doing sports and dieting to
stay healthy, in addition to the great problem posed by bad eating habits in our society and its increase due to
especially due to lack of knowledge about nutrition issues, I have decided to carry out the following project which consists of carrying out a
web application with which all people could obtain an optimized diet according to their characteristics and objectives, in addition to winning
information about nutrition. \\\\

This project will help both people who want to obtain a diet and be able to modify it based on their tastes, needs and economic capacity,
as well as nutritionists who want to create their own diets and increase their knowledge with a large product base in which to compare them
or change them between those that are similar.

\cleardoublepage

\thispagestyle{empty}

\noindent\rule[-1ex]{\textwidth}{2pt}\\[4.5ex]

D. \textbf{Juan Julián Merelo Guervós}, Profesor del Departamento de Arquitectura y Tecnología de Computadores de la Universidad de Granada.

\vspace{0.5cm}

\textbf{Informo:}

\vspace{0.5cm}

Que el presente trabajo, titulado \textit{\textbf{Generación de dietas optimizadas con fines concretos}},
ha sido realizado bajo mi supervisión por \textbf{José Miguel Pelegrina Pelegrina}, y autorizo la defensa de dicho trabajo ante el tribunal
que corresponda.

\vspace{0.5cm}

Y para que conste, expiden y firman el presente informe en Granada a Septiembre de 2021.

\vspace{1cm}

\textbf{El/la director(a)/es: }

\vspace{5cm}

\noindent \textbf{Juan Julián Merelo Guervós}

\chapter*{Agradecimientos}



